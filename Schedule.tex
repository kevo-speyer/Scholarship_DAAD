%% LyX 2.1.3 created this file.  For more info, see http://www.lyx.org/.
%% Do not edit unless you really know what you are doing.
\documentclass[english,12pt]{article}
\usepackage[T1]{fontenc}
\usepackage[latin9]{inputenc}
\usepackage{geometry}
\geometry{verbose,tmargin=2.8cm,bmargin=2.8cm,lmargin=2.6cm,rmargin=2.6cm}
\usepackage{textcomp}
\usepackage{graphicx}
\sloppy

\makeatletter

%%%%%%%%%%%%%%%%%%%%%%%%%%%%%% LyX specific LaTeX commands.
%% Because html converters don't know tabularnewline
\providecommand{\tabularnewline}{\\}

\makeatother

\usepackage{babel}
\begin{document}

\sloppy

\title{Schedule of planned research work\\
{\bf Active polymer brushes}}

\maketitle

\noindent
\begin{tabular}{ll}
Applicant: & Lic.~Kevin Speyer \\
Home institution: & National Atomic Energy Commission, Pcia.~Buenos Aires, Argentina \\
Host institution: & Institut f\"ur Theoretische Physik, Georg-August Universit\"at, G\"ottingen\\
Duration: & 2 months\\
Starting date: & May 2018
\end{tabular}

\section{Schedule}

\begin{itemize}
\item 1 May to 1 June: Channel with explicit solvent: synchronization and hydrodynamic coupling

\begin{itemize}
\item An explicit solvent will be added to channel coated by active brushes,
described by Lennard-Jones particles, to account for momentum conservation
and the concomitant hydrodynamic interactions, taking advantage of
the experience of the German group. 
\item We will analyze the effect of solvent-mediated hydrodynamic coupling
between active chains and characterize changes in the collective chain
dynamics, as compared to the case of elastic coupling alone (in dry
brushes). 
\end{itemize}
\item 1 June to 1 July: Liquid flow with synchronized chain dynamics

\begin{itemize}
\item Imposing coordinated movement to the chains, mimicking typical cilia
dynamics, we will study the flow generated in the solvent. 
\item Upper and lower active brush layers of the slit channel will be studied
as a function of polymer beating frequency, amplitude, and direction.
Directed flow in the vicinity of the individual active brush layers
can be achieved by choosing parameters that result in synchronization, or by
imposing a phase-locked dynamics with a time-dependent external force. 
\item Special interesting cases are in-phase movement of active upper and
lower brushes and anti-phase movement of upper and lower grafted layers.
If the polymers drive locally the fluid, the in-phase movement is
expected to produce a plug flow, whereas the anti-phase movement results
in shear flow. 
\item A parallelization scheme with GPU of some parts of the code will be
studied and implemented by the applicant with the help of the Prof.
M\"uller and his group. 
\end{itemize}
\end{itemize}
%
\end{document}
